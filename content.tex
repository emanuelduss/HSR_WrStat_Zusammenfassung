%%%%%%%%%%%%%%%%%%%%%%%%%%%%%%%%%%%%%%%%%%%%%%%%%%%%%%%%%%%%%%%%%%%%%%%%
\section{Kombinatorik (Zählen)}

\subsection{Produktregel: Die Für–jedes–gibt–es–Regel (FJGE)}
Fur jede der $n_1$ Möglichkeiten gibt es eine von der
ersten Position unabhängige Anzahl $n_2$ Möglichkeiten für den Rest.
\[ = n_1 \cdot n_2 \text{ Möglichkeiten } \]

\subsection{Permutationen: Reihenfolge}
Auf wieviele Arten kann man $n$ Objekte anordnen? (Anzahl Anordnungen)
\[ = n! \text{ Arten} \]

\subsection{Kombinationen: Auswahl}
Auf wieviele Arten kann man $k$ Objekte aus $n$ auswählen?
\[ = C^n_k=\binom{n}{k} = \frac{n!}{k!(n-k)!} \text{ Arten} \]
Dieser "`Binomialkoeffizient"' lässt sich auf dem Taschenrechner
TI-36XII mit \texttt{n nCr k} (unter \texttt{PRB}), mit dem Voyage 200
mit \texttt{nCr(n,k)}, in Sage mit \texttt{binomial(n,k)} und in 
Octave/Matlab mit \texttt{nchoosek(n,k)} berechnen.

Auf wieviele Arten kann man $k$ Mal eine Auswahl aus $n$ Objekten
treffen?
\[ = n^k \text{ Arten} \]

%%%%%%%%%%%%%%%%%%%%%%%%%%%%%%%%%%%%%%%%%%%%%%%%%%%%%%%%%%%%%%%%%%%%%%%%
\section{Ereignisse und Wahrscheinlichkeit}
\subsection{Vorgehensweise}
Man braucht ein Experiment.
\begin{itemize}
  \item Elementarereignisse $\Omega$ sind alle möglichen Versuchsausgänge
  \item Ereignis $A \in \Omega$
  \item Wahrscheinlichkeit dass Ereignis $A$ eintritt ist $P(A)$
\end{itemize}
\subsection{Eigenschaften und Regeln}
\begin{itemize}
  \item $0 \le P(A) \le 1$: Warscheinlichkeit ist immer zwischen $0$
    und $1$
  \item $P(A) < P(B)$: Die Warscheinlichkeit für das Ereignis $A$ ist
    kleiner als für $B$
  \item $P(\Omega) = 1$: Das sichere Ereignis tritt immer ein
  \item $P(\emptyset) = 0$: Das unmöglich Ereignis tritt nie ein
  \item $P(A \cap B)$: Ereignis $A$ und Ereignis $B$ tritt ein
    \begin{itemize}
      \item Falls unabhängig: $= P(A) \cdot P(B)$
      \item Falls abhängig: Nicht alleine aus $P(A)$ und $P(B)$ berechenbar!
    \end{itemize}
  \item $P(A \cup B) = P(A) + P(B) - P(A \cap B)$\footnote{Ein-
    Ausschaltformel}: Ereignis $A$ oder Ereignis $B$ tritt ein
    \begin{itemize}
      \item Falls sich $A$ und $B$ nicht überschneiden\footnote{
        Paarweise disjunkt: $A$ und $B$ treffen nicht gleichzeitig ein}:
        $= P(A) + P(B)$
    \end{itemize}
  \item $P(A \setminus B) = 1 - P(A)$: Ereignis $A$ tritt ein, aber ohne
    Ereignis $B$
  \item $P(\bar{A})$: Ereignis $A$ tritt nicht ein
\end{itemize}
